\documentclass[11p]{article}
% Packages
\usepackage{amsmath}
\usepackage{graphicx}
\usepackage[swedish]{babel}
\usepackage[
    backend=biber,
    style=authoryear-ibid,
    sorting=ynt
]{biblatex}
\usepackage[utf8]{inputenc}
\usepackage[T1]{fontenc}
\usepackage{blindtext}
\usepackage{siunitx}
%Källor

\graphicspath{ {./images/} }

\title{Labrapprt \\ \small Fysik 1}
\author{Magnus Silverdal }
\date{\today}

\begin{document}

    \begin{titlepage}
        \begin{center}
            \vspace*{1cm}

            \Huge
            \textbf{Laboration 5}

            \vspace{0.5cm}
            \LARGE
            Ellära

            \vspace{1.5cm}

            \textbf{Linus Lundqvist}

            \vfill


            Fysik 1

            \vspace{0.8cm}

            \includegraphics[width=0.4\textwidth]{NTI Gymnasiet_Symbol_print_svart.png}

            \Large
            Teknikprogrammet\\
            NTI Gymnasiet\\
            Umeå\\
            \today

        \end{center}
    \end{titlepage}
    \section{Syfte och frågeställning}

    Syftet med laborationen var att ta reda på blah blah blah

    \section{Material}

    \begin{itemize}
        \item 1 - Batteri
        \item 1 - Multimeter
        \item 5 - kablar
        \item 2 - kopplingssplint
        \item 2 - Resistor, $100\Omega$
    \end{itemize}

    \section{Del 1}

    \subsection{Metod}
    Utöver denna uppgift skall man koppla ihop

    \subsection{Resultat}
    0,03V \\
    0,0003A \\

    \subsection{Analys}


    \section{Del 2}

    \subsection{Metod}
    Utöver denna uppgift skall man koppla ihop

    \subsection{Resultat}
        0,015V \\
        0,0003A \\

    \subsection{Analys}
    a

    \section{Del 3}
    \subsection{Metod}
    Utöver denna uppgift skall man koppla ihop

    \subsection{Resultat}
        0,03V \\
        0,0003A \\

    \subsection{Analys}


    \section{Diskussion}


\end{document}
